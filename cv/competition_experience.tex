%-------------------------------------------------------------------------------
%	SECTION TITLE
%-------------------------------------------------------------------------------
\cvsection{COMPETITION EXPERIENCE}


%-------------------------------------------------------------------------------
%	CONTENT
%-------------------------------------------------------------------------------
\begin{cventries}

  \cventry
    {}
    {Market Segment Sales Predication Model of CCF BDCI Passenger Vehicles}
    {Beijing, China}
    {09/2019 - 11/2019}
    {
      \textbf{Role:} Leader
      \newline
      \textbf{Award:} Top 3\%
      \newline
      \textbf{Objective:} To predict the sales volume in the next 4 months of each province and vehicle module
      \begin{itemize}
        \item {Analyzed the sales of 60 vehicle models in 22 provinces, and the search volume of each vehicle model in each province in the past 4 months}
        \item {Used sliding window to build feature, and calculate the correlation rate among variables and then remove low-correlation variable from feature}
        \item {Divided trained data into training set and testing set}
        \item {Used lstm, xgb, lgb models to train data and get the sales of next month}
      \end{itemize}
    }
  
  \cventry
    {}
    {Fengru Cup Science and Technology Contest of Beihang University}
    {Beijing, China}
    {11/2018 - 05/2019}
    {
      \textbf{Project:} Automatic Annotation Generation Software for Java \quad \textbf{Advisor:} Xu Wang
      \newline
      \textbf{Role:} Leader
      \newline
      \textbf{Award:} Third Award
      \newline
      \textbf{Description:} Annotations are automatically generated for Java through deep learning nlp programs, and integrated as a plugin for java IDE IDEA, which provides the function of automatically generating comments after IDEA user code
      \begin{itemize}
        \item {Used web crawler to acquire 20G data of Java codes and annotation from Github by start rating for training}
        \item {Conducted data processing with Python, paired data by python, and separated training data from testing data}
        \item {Read relative papers, reproduced sequence2sequence model using pytorch}
        \item {Trained this model with training data on a distributed cluster to train fast}
        \item {Applied Java to package a plugin for IDEA, which could realize the automatic annotation}
      \end{itemize}
    }

%---------------------------------------------------------
\end{cventries}
