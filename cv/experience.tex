%-------------------------------------------------------------------------------
%	SECTION TITLE
%-------------------------------------------------------------------------------
\cvsection{Experience}


%-------------------------------------------------------------------------------
%	CONTENT
%-------------------------------------------------------------------------------
\begin{cventries}

  \cventry
    {见习深度学习工程师} % Job title
    {商汤} % Organization
    {深圳, 中国} % Location
    {3. 2020 - 9. 2020} % Date(s)
    {
      \begin{itemize} % Description(s) of tasks/responsibilities
        \item {参与设计并实现深度学习pipeline框架,调用各个功能组件实现数据处理,训练,评测,模型转换等stage的pipeline。}
          \begin{itemize}
            \item {实现了各个框架分布式训练接口的统一,包括pytorch,单机训练,以及公司内部框架。}
            \item {设计并实现了各个stage输入参数的规范,支持单个string类型,path类型,list类型。}
            \item {完成了框架的sphinx文档。}
            \item {完成了框架的单元测试代码。}
          \end{itemize}
        \item {完成组内docker的基本构建。}
          \begin{itemize}
            \item {完成docker镜像的构建,支持深度学习的基本环境,包括gcc,pytorch,cuda,和其他依赖。}
            \item {在服务器上配置了nvidia-docker,支持container内使用gpu。}
            \item {为分类,检测等各个组件打包了docker image,支持从osg读取训练数据。}
          \end{itemize}
        \item {完成组内gitlab-CI的基本构建。}
          \begin{itemize}
            \item {在服务器上配置gitlab-runner。}
            \item {完成CI的配置文件,支持在CI中进行,风格检查,调用组件打包好的docker单元测试,文档部署,docker镜像打包。}
          \end{itemize}
        \item {阅读mimic相关paper,并在组件中进行复现和支持。}
        \item {工具的对接,使用mimic相关组件为tracking组件提高准确率。}
      \end{itemize}
    }
  
  \cventry
    {本科实习生}
    {北航ACT实验室-糖尿病药物推荐}
    {北京,中国}
    {12. 2018 - 5. 2019}
    {
      \begin{itemize}
        \item {阅读相关paper,了解相关方法进行整理。}
        \item {实现医院数据parse的脚本,对于部分数据采用处理表格等库进行处理,对于难以处理的数据采用ocr进行辅助处理,将数据处理为 血糖-时间点-药物。}
        \item {对于数据采用xgb等机器学习模型进行训练,实现给定时间点,给定药物的血糖的预测。}
      \end{itemize}
    }

  \cventry
    {组长,初赛成绩35/2500,进入复赛}
    {datafountain乘用车销量预测比赛}
    {}
    {10. 2019 - 12. 2019}
    {
      \begin{itemize}
        \item {对分省,分车型汽车销量的数据进行特征工程。}
          \begin{itemize}
            \item {采用最小二乘法计算各个特征与预测值的相关性,去掉不相关或相关性小的特征。}
            \item {采用滑窗构建多个时序特征。}
          \end{itemize}
        \item {采用多种模型进行训练,包括xgb,lgb,lstm等。}
        \item {对多种模型进行融合。}
      \end{itemize}
    }

  \cventry
    {组长,三等奖}
    {北航冯如杯科技竞赛-java代码方法级别注释的自动生成}
    {}
    {11. 2018 - 4. 2019}
    {
      \begin{itemize}
        \item {在github上爬去java代码。}
        \item {对爬去的代码进行处理为方法-注释对,同时把单词进行词性归一,降低词典大小。}
        \item {使用pytorch实现seq2seq模型。}
        \item {对代码和注释采用embeding编码进行训练,实现代码到注释的翻译。}
        \item {将模型封装为IDEA的插件,能够实现在IDEA内对选中的方法生成注释。}
      \end{itemize}
    }


%---------------------------------------------------------
\end{cventries}
